%!TEX root = handout_precourse.tex


% side rules for theorem environment


\newmdenv[
  nobreak = true,
  linewidth=.3pt,
  linecolor=gray!80!white,
  topline=false,
  bottomline=false,
  leftmargin=0pt, 
  innerleftmargin=0.4em, 
  rightmargin=0pt, 
  innerrightmargin=0pt, 
  innertopmargin=-5pt ,
  innerbottommargin=3pt, 
  splittopskip=\topskip, 
  splitbottomskip=0.3\topskip,
  backgroundcolor=white!4,
  rightline=false,
]{siderules}


\newmdenv[
  nobreak = true,
  linewidth=.3pt,
  linecolor=black!80!white,
  topline=false,
  bottomline=false,
  leftmargin=0pt, 
  innerleftmargin=0.4em, 
  rightmargin=0pt, 
  innerrightmargin=0pt, 
  innertopmargin=-5pt ,
  innerbottommargin=3pt, 
  splittopskip=\topskip, 
  splitbottomskip=0.3\topskip,
  backgroundcolor= gray!5!white,
  rightline=false,
]{siderules2}



%Theorem, Corollary, Proposition, Lemma

\newtheoremstyle{theostyle}
  {} % Space above
  {} % Space below
  {} % Body font
  {} % Indent amount
  {{\hspace{-.2cm} \color{red!30!black!}\ding{69}}~\itshape\bfseries\color{red!30!black!}} % Theorem head font
  {.} % Punctuation after theorem head
  {.5em} % Space after theorem head
  {} % Theorem head spec (can be left empty, meaning `normal')
\theoremstyle{theostyle} 
\newtheorem{theorem}{Theorem}[chapter] %%% changed
\newtheorem*{theorem*}{Theorem} %%% changed
\newtheorem{corollary}{Corollary}[chapter] %%% changed
\newtheorem{proposition}{Proposition}[chapter] %%% changed
\newtheorem{lemma}{Lemma}[chapter] %% chaged
% \newtheorem{hyp}[theorem]{Hypothesis} %%% changed

%--------------------------------

%Defintion, Assumptions

\definecolor{darkblue}{rgb}{0.0, 0.0, 0.55}
\newtheoremstyle{defnstyle}
  {} % Space above
  {} % Space below
  {} % Body font
  {} % Indent amount
  {{\hspace{-.2cm} \color{darkblue!40!black}\ding{118}}~\itshape\bfseries\color{darkblue!40!black}} %  head font
  {.} % Punctuation after  head
  {.5em} % Space after head
  {} %  head spec (can be left empty, meaning `normal')
\theoremstyle{defnstyle} 
\newtheorem{definition}{Definition}[chapter] 
\newtheorem{assump}[theorem]{Assumption} 


% Examples,
\definecolor{darkgoldenrod}{rgb}{0.72, 0.53, 0.04}
\newtheoremstyle{exmstyle}
  {} % Space above
  {} % Space below
  {} % Body font
  {} % Indent amount
  {{\hspace{-.1cm}\color{darkgoldenrod!60!black} \ding{168}}~\bfseries\color{darkgoldenrod!60!black}} %  head font
  {.} % Punctuation after  head
  {.5em} % Space after head
  {} %  head spec (can be left empty, meaning `normal')
\theoremstyle{exmstyle}
\newtheorem{example}{Example}[chapter] %%% changed

% Remark, Remarks,
  \definecolor{glaucous}{rgb}{0.38, 0.51, 0.71}
\newtheoremstyle{remstyle}
  {} % Space above
  {} % Space below
  {} % Body font
  {} % Indent amount
  {{\hspace{-.55cm}  \color{glaucous!30!black} \ding{46}}~\bfseries\color{glaucous!30!black}} %  head font
  {.} % Punctuation after  head
  {.5em} % Space after head
  {} %  head spec (can be left empty, meaning `normal')
\theoremstyle{remstyle}
\newtheorem{remark}{Remark}[chapter] %%% changed
\newtheorem{remarks}{Remarks}[chapter] %%% changed



% \newtheorem*{theorem*}{Theorem}
\newtheorem*{claim}{Claim}

%notation

\newtheoremstyle{notation}
  {} % Space above
  {} % Space below
  {} % Body font
  {} % Indent amount
  {{\hspace{-.75cm}  \color{gray!10!black!} \lhdbend}~\bfseries\color{gray!80!black!}} %  head font
  {:} % Punctuation after  head
  {.5em} % Space after head
  {} %  head spec (can be left empty, meaning `normal')
\theoremstyle{notation}
\newtheorem*{notation}{{Notes on notations}}


\AtBeginEnvironment{theorem}{\begin{minipage}{\textwidth}\noindent\vspace*{.2cm}}
\AtEndEnvironment{theorem}{\end{minipage}\vspace*{0cm}\flushright \color{red!30!black!}\textminus\textminus\textminus \ding{69}  }

\AtBeginEnvironment{lemma}{\begin{minipage}{\textwidth}\vspace*{.3cm}}
\AtEndEnvironment{lemma}{\end{minipage}\vspace*{.3cm}}

\AtBeginEnvironment{definition}{\begin{minipage}{\textwidth}\noindent\vspace*{.2cm}}
\AtEndEnvironment{definition}{\end{minipage}\vspace*{0cm}\flushright \color{darkblue!40!black}\textminus\textminus\textminus \ding{118} }




% \AtBeginEnvironment{assumption_n}{\begin{minipage}{\textwidth}\vspace*{.3cm}}
% \AtEndEnvironment{assumption_n}{\vspace*{-27pt}\flushright{\trpslash}\end{minipage}\vspace*{.3cm}}

% \AtBeginEnvironment{assumption}{\begin{minipage}{\textwidth}\vspace*{.3cm}}
% \AtEndEnvironment{assumption}{\vspace*{-27pt}\flushright{\trpslash}\end{minipage}\vspace*{.3cm}}

% \AtBeginEnvironment{example}{\vspace*{.3cm}}
\AtEndEnvironment{example}{\color{darkgoldenrod!60!black!} \flushright{\textit{(...end of example!)}}}


%% numbering of theorems



% changing the numbering of theorem and equations
% \numberwithin{section}{chapter}
% \numberwithin{equation}{chapter}
% \counterwithin{equation}{chapter}


% color brackets
\makeatletter
\newcount\bracketnum
\newcommand\makecolorlist[1]{%
    \bracketnum0\relax
    \makecolorlist@#1,.%
    \bracketnum0\relax
}
\def\makecolorlist@#1,{%
    \advance\bracketnum1\relax
    \expandafter\def\csname bracketcolor\the\bracketnum\endcsname{\color{#1}}%
    \@ifnextchar.{\@gobble}{\makecolorlist@}%
}
\let\oldleft\left
\let\oldright\right
\def\left#1{%
    \global\advance\bracketnum1\relax 
    \colorlet{temp}{.}%
    \csname bracketcolor\the\bracketnum\endcsname
    \oldleft#1%
    \color{temp}%
}
\def\right#1{%
    \colorlet{temp}{.}%
    \csname bracketcolor\the\bracketnum\endcsname
    \oldright#1%
    \global\advance\bracketnum-1\relax
    \color{temp}%
}
\makeatother


\makecolorlist{black,blue,red}




% 2 independent signs

% indep one
\newcommand{\indep}{\raisebox{0.05em}{\,\rotatebox[origin=c]{90}{\mbox{\Large$\models$}} \,} }
\newcommand{\indepG}{{\indep}_{\mbox{\tiny$\mkern-18mu\mathcal{G}$}}}

% independent one
\newcommand\independent{\protect\mathpalette{\protect\independenT}{\mbox{\Large$\perp$}}}
\def\independenT#1#2{\mathrel{\rlap{$#1#2$}\mkern2mu{#1#2}}}

\newcommand{\independentG}{{\independent}_{\mbox{\scriptsize$\mkern-6mu\mathcal{G}$}}}

% argmin, argmax
\DeclareMathOperator*{\argmax}{arg\,max}
\DeclareMathOperator*{\argmin}{arg\,min}


% combination notation

\newcommand\Myperm[2][^n]{\prescript{#1\mkern-2.5mu}{}P_{#2}}
\newcommand\Mycomb[2][^n]{\prescript{#1\mkern-0.5mu}{}C_{#2}}




% distributed as
\makeatletter
\newcommand{\distas}[1]{\mathbin{\overset{#1}{\kern\z@\sim}}}%
\newsavebox{\mybox}\newsavebox{\mysim}
\newcommand{\distras}[1]{%
  \savebox{\mybox}{\hbox{\kern3pt$\scriptstyle#1$\kern3pt}}%
  \savebox{\mysim}{\hbox{$\sim$}}%
  \mathbin{\overset{#1}{\kern\z@\resizebox{\wd\mybox}{\ht\mysim}{$\sim$}}}%
}
\makeatother


% tikz

\pgfarrowsdeclare{my to}{my to}
{
  \pgfarrowsleftextend{-2\pgflinewidth}
  \pgfarrowsrightextend{\pgflinewidth}
}
{
  \pgfsetlinewidth{1.4\pgflinewidth}
  \pgfsetdash{}{0pt}
  \pgfsetroundcap
  \pgfsetroundjoin
  \pgfpathmoveto{\pgfpoint{-5.5\pgflinewidth}{7.5\pgflinewidth}}
  \pgfpathcurveto
  {\pgfpoint{-4.0\pgflinewidth}{0.1\pgflinewidth}}
  {\pgfpoint{0pt}{0.25\pgflinewidth}}
  {\pgfpoint{0.75\pgflinewidth}{0pt}}
  \pgfpathcurveto
  {\pgfpoint{0pt}{-0.25\pgflinewidth}}
  {\pgfpoint{-4.0\pgflinewidth}{-0.1\pgflinewidth}}
  {\pgfpoint{-5.5\pgflinewidth}{-7.5\pgflinewidth}}
  \pgfusepathqstroke
}


\newcommand{\asymcloud}[2][.1]{%
\begin{scope}[#2]
\pgftransformscale{#1}%    
\pgfpathmoveto{\pgfpoint{261 pt}{115 pt}} 
  \pgfpathcurveto{\pgfqpoint{70 pt}{107 pt}}
                 {\pgfqpoint{137 pt}{291 pt}}
                 {\pgfqpoint{260 pt}{273 pt}} 
  \pgfpathcurveto{\pgfqpoint{78 pt}{382 pt}}
                 {\pgfqpoint{381 pt}{445 pt}}
                 {\pgfqpoint{412 pt}{410 pt}}
  \pgfpathcurveto{\pgfqpoint{577 pt}{587 pt}}
                 {\pgfqpoint{698 pt}{488 pt}}
                 {\pgfqpoint{685 pt}{366 pt}}
  \pgfpathcurveto{\pgfqpoint{840 pt}{192 pt}}
                 {\pgfqpoint{610 pt}{157 pt}}
                 {\pgfqpoint{610 pt}{157 pt}}
  \pgfpathcurveto{\pgfqpoint{531 pt}{39 pt}}
                 {\pgfqpoint{298 pt}{51 pt}}
                 {\pgfqpoint{261 pt}{115 pt}}
\pgfusepath{fill,stroke}         
\end{scope}}  


%
\def\FunctionF(#1){(#1)^3- 3*(#1)}%

% \tikzset{-{{To}[length=1.3mm,line width=1.3pt]}, auto, 
%     state/.style ={circle, draw, minimum width = 0.1 cm, scale=0.9},
%     point/.style = {circle, draw, inner sep=0.02cm,fill,node contents={}},
%     bidirected/.style={Latex-Latex,dashed},
%     hidden/.style={dashed}
%     }

\newcommand{\tikzl}[1]{{\begin{tikzpicture}#1\end{tikzpicture}}}

\newcommand{\tikzm}[1]{\footnotesize{\begin{tikzpicture}#1\end{tikzpicture}}}


% qed symbol
\renewcommand{\qedsymbol}{\rule{0.4em}{0.4em}}


\newcommand{\trpslash}{\textcolor{green!35!black}{\textbackslash\hspace{-.15cm}\textbackslash\hspace{-.15cm}\textbackslash   }}

\newcommand{\lgreen}{green!15!white}

%writing hand

\newcommand\Writinghand{{\fontfamily{mvs}\fontencoding{U}\selectfont\char98}}



%% argmin
\DeclareMathOperator*{\argminNS}{argmin}  
\DeclareMathOperator*{\argmaxNS}{argmin}  


%--------------------- added by Maxime


%%%%%%%%%%%%%%%%%%%%%%%%%%%%%%%%%%%%%%%%%%%%%%%%%%%%%%%%%%%%%%%%%%%%%%%%%%%%%%%%
%% Theromstil anpassen
%%%%%%%%%%%%%%%%%%%%%%%%%%%%%%%%%%%%%%%%%%%%%%%%%%%%%%%%%%%%%%%%%%%%%%%%%%%%%%%%
% \declaretheoremstyle[
% spaceabove = \topsep,
% spacebelow = \topsep,
% headfont = \normalfont\bfseries,
% bodyfont = \normalfont,
% headindent = \parindent,
% postheadspace = \newline,
% headpunct = {}]{definition}

% \declaretheorem[style = definition, numberwithin = chapter, qed = $\blacktriangle$]{definition}
% \declaretheorem[style = definition, numberwithin = chapter, qed = $\blacksquare$]{theorem}
% \declaretheorem[style = definition, numberwithin = chapter, qed = $\blacktriangledown$]{comment}
% \declaretheorem[style = definition, numberwithin = chapter, qed = $\bigstar$]{example}
% \declaretheorem[style = definition, numberwithin = chapter, qed = $\blacksquare$]{korollar}
% \declaretheorem[style = definition, numberwithin = chapter, qed = $\blacksquare$]{lemma}


%%%%%%%%%%%%%%%%%%%%%%%%%%%%%%%%%%%%%%%%%%%%%%%%%%%%
%%%%%%%%%%%%%%%% Further definitions %%%%%%%%%%%%%%%
%%%%%%%%%%%%%%%%%%%%%%%%%%%%%%%%%%%%%%%%%%%%%%%%%%%%
%

\newfont{\suet}{suet14 at 20pt}
%
%\renewcommand{\labelenumi}{\alph{enumi})}
%
\newcommand{\R}{\mathbb{R}}
\newcommand{\Q}{\mathbb{Q}}
\newcommand{\Z}{\mathbb{Z}}
\newcommand{\N}{\mathbb{N}}
\newcommand{\Com}{\mathbb{C}}

\newcommand{\Cov}{\text{Cov}}
\newcommand{\Cor}{\text{Cor}}
\newcommand{\E}{\text{E}}
\newcommand{\Var}{\text{Var}}
\newcommand{\Bias}{\text{Bias}}
\newcommand{\MSE}{\text{MSE}}
\newcommand{\FI}{\text{FI}}
\newcommand{\KI}{\text{KI}}
\newcommand{\LQ}{\text{LQ}}
\newcommand{\SXY}{\text{SXY}}
\newcommand{\SSX}{\text{SSX}}
\newcommand{\SSA}{\text{SSA}}
\newcommand{\SSB}{\text{SSB}}
\newcommand{\SSE}{\text{SSE}}
\newcommand{\SSG}{\text{SSG}}
\newcommand{\SST}{\text{SST}}
\newcommand{\MSA}{\text{MSA}}
\newcommand{\MSB}{\text{MSB}}
\newcommand{\MST}{\text{MST}}

\newcommand{\sign}{\text{sign}}
\newcommand{\Rang}{\text{Rang}}

%
%\newcommand{\fX}{\mathfrak{X}}
%\newcommand{\fM}{\mathfrak{M}}
%\newcommand{\fA}{\mathfrak{A}}
%\newcommand{\fE}{\mathfrak{E}}
%\newcommand{\fC}{\mathfrak{C}}
%
\newcommand{\cA}{\mathcal{A}}
\newcommand{\cC}{\mathcal{C}}
\newcommand{\cT}{\mathcal{T}}
\newcommand{\cL}{\mathcal{L}}
\newcommand{\cR}{\mathcal{R}}
\newcommand{\cB}{\mathcal{B}}
\newcommand{\cF}{\mathcal{F}}
\newcommand{\cK}{\mathcal{K}}
\newcommand{\cV}{\mathcal{V}}
\newcommand{\cN}{\mathcal{N}}
\newcommand{\cP}{\mathcal{P}}
\newcommand{\cD}{\mathcal{D}}
\newcommand{\cE}{\mathcal{E}}
\newcommand{\cM}{\mathcal{M}}
\newcommand{\cX}{\mathcal{X}}
\newcommand{\Ahr}{\tilde{\mathcal{A}}_r}
%
\newcommand{\sB}{\mbox{\suet B}}
\newcommand{\sE}{\mbox{\suet E}}
\newcommand{\sA}{\mbox{\suet A}}
\newcommand{\sC}{\mbox{\suet C}}
\newcommand{\sF}{\mbox{\suet F}}
\newcommand{\sX}{\mbox{\suet X}}
\newcommand{\sD}{\mbox{\suet D}}
%
%\newcommand{\ccA}{{\cal A}}
%\newcommand{\ccC}{{\cal C}}
%\newcommand{\ccL}{{\cal L}}
%
%
\newcommand{\Ra}{\Rightarrow}
\newcommand{\ra}{\rightarrow}
\newcommand{\LRa}{\Leftrightarrow}
\newcommand{\es}{\emptyset}
\newcommand{\ds}{\displaystyle}
\newcommand{\oA}{\overline{A}}
\newcommand{\oB}{\overline{B}}
\newcommand{\sm}{\setminus}
%
\newcommand{\wraum}{\mbox{$(\Omega,\cA,P)$ }}
%
%\newcommand{\dX}{\mbox{\bf X}}
%\newcommand{\dI}{\mbox{\bf I}}
%\newcommand{\dA}{\mbox{\bf A}}
%\newcommand{\dY}{\mbox{\bf Y}}
%\newcommand{\dB}{\mbox{\bf B}}
%\newcommand{\dM}{\mbox{\bf M}}
%\newcommand{\va}{\mbox{\bf a}}
%
\newcommand{\vmu}{\mbox{\boldmath{$\mu$}}}
\newcommand{\vL}{\mbox{\boldmath{$\Lambda$}}}
\newcommand{\vx}{\mbox{\boldmath{$x$}}}
\newcommand{\vX}{\mbox{\boldmath{$X$}}}
\newcommand{\vbeta}{\mbox{\boldmath{$\beta$}}}

%
%\newcommand{\ta}{\tilde{a}}
%\newcommand{\tb}{\tilde{b}}
%
% \newcommand{\norm}{\mathcal{N}(\mu,\sigma^2)}
% \newcommand{\snorm}{\mathcal{N}(0,1)}
% \newcommand{\Pfolge}{(P_n:n\in \N)\subset \cP}
% \newcommand{\Xfolge}{X_1,\hdots , X_n}
% \newcommand{\konv}{\overset{n\ra \infty}{\longrightarrow}}
%\newcommand{\statraum}{\mbox{$(\X,\sX,\cP)$ }}
%
%
\newcommand{\bfA}{\textbf{A}}
\newcommand{\bfB}{\textbf{B}}
\newcommand{\bfC}{\textbf{C}}
\newcommand{\bfc}{\textbf{c}}
\newcommand{\bfe}{\textbf{e}}
\newcommand{\bfI}{\textbf{I}}
\newcommand{\bfV}{\textbf{V}}
\newcommand{\bfW}{\textbf{W}}
\newcommand{\bfX}{\textbf{X}}
\newcommand{\bfx}{\textbf{x}}
\newcommand{\bfY}{\textbf{Y}}
\newcommand{\bfy}{\textbf{y}}

%
%
\newcommand{\ld}{\lambda} 
%\newcommand{\om}{\omega}
\newcommand{\vt}{\vartheta}
\newcommand{\Tt}{\Theta}
\newcommand{\vp}{\varphi}
\newcommand{\eps}{\epsilon}
\newcommand{\tr}{\text{tr}}
%
%\newcommand{\ud}{\underline}
%
%\newcommand{\bi}{\begin{itemize}}
%\newcommand{\ei}{\end{itemize}}
%\newcommand{\ba}{\begin{align*}}
%\newcommand{\eal}{\end{align*}}
%\newcommand{\ben}{\begin{enumerate}}
%\newcommand{\een}{\end{enumerate}}
%
%
\newcommand{\Xis}{X_1, ..., X_n}
\newcommand{\xis}{x_1, ..., x_n}
\newcommand{\TX}{T_{\vt}(X_1, ..., X_n)}
\newcommand{\TsX}{T_{\vt}'(X_1, ..., X_n)}
\newcommand{\TssX}{T_{\vt}''(X_1, ..., X_n)}
\newcommand{\TstX}{T_{\vt}^*(X_1, ..., X_n)}
\newcommand{\vts}{\vt_1, ..., \vt_k}

\newcommand{\dx}{\mathrm{d}x}
\newcommand{\dy}{\mathrm{d}y}


%align space

\newcommand{\zerodisplayskips}{%
  \setlength{\abovedisplayskip}{-10pt}%
  \setlength{\belowdisplayskip}{8pt}%
  \setlength{\abovedisplayshortskip}{0pt}%
  \setlength{\belowdisplayshortskip}{0pt}}
\appto{\normalsize}{\zerodisplayskips}
\appto{\small}{\zerodisplayskips}
\appto{\footnotesize}{\zerodisplayskips}